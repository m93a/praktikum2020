% !TEX program = xelatex

% Základní balíčky
\documentclass[10pt,a4paper]{article}
\usepackage[utf8]{inputenc}
\usepackage[T1]{fontenc}
\usepackage{graphicx}
\usepackage{wrapfig}
\usepackage{nonfloat}
\usepackage{amsmath}
\usepackage{hyperref}
\usepackage{gensymb}
\usepackage[top = 1cm, bottom = 1cm, left = 1cm, right = 1cm]{geometry}

% Language-related
\usepackage[czech]{babel}
\usepackage{csquotes}
\usepackage{polyglossia}
\setmainlanguage{czech}
\setotherlanguage{greek}

% Bibtex je oficiálně mrtka
% \usepackage[backend=bibtex,style=verbose-trad2]{biblatex}
% \usepackage{etoolbox}
% \patchcmd{\thebibliography}{\section*{\refname}}{}{}{}
% \bibliography{protokol}

% Pro titulní stránku
\usepackage{titlesec}
\usepackage{setspace}
\usepackage{framed}
\usepackage{array}

% Vlastní balíčky
\usepackage{gnuplottex}
\usepackage{epstopdf}
\usepackage{csvsimple}
\usepackage{units}

\usepackage{soul}




\renewcommand{\U}[1]{\ensuremath{\,\mathrm{#1}}}
\newcommand{\°}{\degree}

\newcommand{\titjmeno}{Michal Grňo}
\newcommand{\titobor}{FOF}


\newcommand{\titcislo}{D5}
\newcommand{\titnazev}{Spektrometrie alfa}
\newcommand{\titmereni}{\st{nikdy}}
\newcommand{\titodevzdani}{8. 11. 2020}


\begin{document}

\include{titulka}
\setmainfont{Linux Libertine O}




\section{Pracovní úkoly}
\begin{enumerate}
    \item Naměřte číslo tři.
\end{enumerate}

\section{Teoretická část}
Měli bychom naměřit 3.

\section{Výsledky měření}
Naměřili jsme $2,8 \pm 0,3$.


\section{Diskuse}
Bylo to ok? Jo.

\section{Závěr}
Pěkný 👍️


\section{Literatura}
[1] Nosek D. a J. Vrzal. Spektrometrie záření α. Dostupné z: \url{https://physics.mff.cuni.cz/vyuka/zfp/_media/zadani/texty/txt_405.pdf}. 30. září 20031
\\
\st{[2] Nesnáším BibTeX.}


\end{document}