% !TEX program = xelatex

% Základní balíčky
\documentclass[10pt,a4paper]{article}
\usepackage[utf8]{inputenc}
\usepackage[T1]{fontenc}
\usepackage{graphicx}
\usepackage{wrapfig}
\usepackage{nonfloat}
\usepackage{amsmath}
\usepackage{amssymb}
\usepackage{mathtools}
\usepackage{hyperref}
\usepackage{gensymb}
\usepackage[top = 1cm, bottom = 1cm, left = 1cm, right = 1cm]{geometry}

% Language-related
\usepackage[czech]{babel}
\usepackage{csquotes}
\usepackage{polyglossia}
\setmainlanguage{czech}
\setotherlanguage{greek}

% Bibtex je oficiálně mrtka
% \usepackage[backend=bibtex,style=verbose-trad2]{biblatex}
% \usepackage{etoolbox}
% \patchcmd{\thebibliography}{\section*{\refname}}{}{}{}
% \bibliography{protokol}

% Pro titulní stránku
\usepackage{titlesec}
\usepackage{setspace}
\usepackage{framed}
\usepackage{array}

% Vlastní balíčky
\usepackage{gnuplottex}
\usepackage{epstopdf}
\usepackage{csvsimple}
\usepackage{units}
\usepackage{subfig}
\usepackage{pdfpages}
\usepackage{multirow}

\usepackage{soul}

\usepackage{calc}
\newcommand*{\mask}[2]{\mathord{\makebox[\widthof{\(#1\)}]{\(#2\)}}}




\renewcommand{\U}[1]{\ensuremath{\,\mathrm{#1}}}
\newcommand{\°}{\degree}

\newcommand{\titjmeno}{Michal Grňo}
\newcommand{\titobor}{FOF}


\newcommand{\titcislo}{A19}
\newcommand{\titnazev}{Rentgenografické difrakční určení mřížového parametru známé kubické látky}
\newcommand{\titmereni}{19. 11. 2020}
\newcommand{\titodevzdani}{4. 12. 2020}


\renewcommand{\t}[1]{\mathrm{#1}}


\begin{document}

\include{titulka}
\setmainfont{Linux Libertine O}




\section{Pracovní úkoly}
\begin{enumerate}

    \item \st{Nalezněte standardní rtg práškový difraktogram v databázi PDF-2 na CD-ROM.}
    \item \st{Určete vhodný úhlový obor měření.}
    \item \st{Připravte vzorek pro měření a proveďte měření na komerčním práškovém difraktometru.}
    \item V průběhu měření zpracujte data dodaná z měření na stejném (obdobném) vzorku provedená většinou předcházející skupinou – nalezněte polohy difrakčních maxim
    \item Z Braggovy rovnice vypočtěte mezirovinné vzdálenosti a mřížové parametry pro jednotlivé difraktující roviny.
    \item Proveďte korekci na instrumentální efekty a určete mřížový parametr zadané kubické látky s maximální přesností.
    \item Diskutujte odchylky mezi určeným parametrem konkrétního vzorku a tabelovaným mřížovým parametrem.

\end{enumerate}

\section{Teoretická část}

\begin{figure}[p]
    \centering
    \begin{gnuplot}[terminal=epslatex,terminaloptions={color size 18cm, 8cm}]
        plot 'zprac1/'.file.'.png_raw.dat' w l t 'jas', 'zprac2/'.file.'.png_peaks.dat' using 1:2 w p lc rgb "red" pt 7 t 'peaky', 'zprac2/'.file.'.png_peaks.dat' using 1:($2+5):(($4 == 2 ? 'x' : $4 == 3 ? 'y' : 'z').sprintf('_%1d',$3)) w labels not
    \end{gnuplot}
    \caption{}
    \label{}
\end{figure}

\section{Diskuse}



\section{Závěr}

\section{Literatura}
[1] Praktikum částicové a jaderné fyziky. Objevování částic v detektoru ATLAS v CERN. \\ Dostupné z: \url{https://physics.mff.cuni.cz/vyuka/zfp/_media/zadani/texty/txt_401.pdf}. 26. září 2019.
\\\\
{}[2] DANIŠ, Stanislav. \textit{Atomová fyzika a elektronová struktura látek.} Praha: MatfyzPress, 2019. \\ ISBN 978-80-7378-376-1. Kapitola Struktura pevných látek.
\\\\
{}[3] SWANSON, H.E. and E. Tatge. \textit{Standard X-ray Difraction Powder Patterns.} National Bureau of Standards. 1953.

\end{document}
