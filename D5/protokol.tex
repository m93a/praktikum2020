% !TEX program = xelatex

% Základní balíčky
\documentclass[10pt,a4paper]{article}
\usepackage[utf8]{inputenc}
\usepackage[T1]{fontenc}
\usepackage{graphicx}
\usepackage{wrapfig}
\usepackage{nonfloat}
\usepackage{amsmath}
\usepackage{hyperref}
\usepackage{gensymb}
\usepackage[top = 1cm, bottom = 1cm, left = 1cm, right = 1cm]{geometry}

% Language-related
\usepackage[czech]{babel}
\usepackage{csquotes}
\usepackage{polyglossia}
\setmainlanguage{czech}
\setotherlanguage{greek}

% Bibtex je oficiálně mrtka
% \usepackage[backend=bibtex,style=verbose-trad2]{biblatex}
% \usepackage{etoolbox}
% \patchcmd{\thebibliography}{\section*{\refname}}{}{}{}
% \bibliography{protokol}

% Pro titulní stránku
\usepackage{titlesec}
\usepackage{setspace}
\usepackage{framed}
\usepackage{array}

% Vlastní balíčky
\usepackage{gnuplottex}
\usepackage{epstopdf}
\usepackage{csvsimple}
\usepackage{units}

\usepackage{soul}




\renewcommand{\U}[1]{\ensuremath{\,\mathrm{#1}}}
\newcommand{\°}{\degree}

\newcommand{\titjmeno}{Michal Grňo}
\newcommand{\titobor}{FOF}


\newcommand{\titcislo}{A21}
\newcommand{\titnazev}{Studium rentgenových spekter}
\newcommand{\titmereni}{11. 11. 2019}
\newcommand{\titodevzdani}{17. 11. 2019}


\begin{document}

\include{titulka}
\setmainfont{Linux Libertine O}




\section{Pracovní úkoly}
\begin{enumerate}
    \item Proveďte energetickou kalibraci α-spektrometru a určete jeho rozlišení.
    \item Určete absolutní aktivitu kalibračního radioizotopu $^{241}$Am.
    \item Změřte závislost ionizačních ztrát α-částic na tlaku vzduchu $\Delta T = \Delta T(p)$.
    \item Určete specifické ionizační ztráty α-částic ve vzduchu při normálním tlaku $-dT/dx = f(T)$. Srovnejte tuto závislost se závislostí získanou pomocí empirické formule pro dolet α-částic ve vzduchu za normálních podmínek.
    \item Určete energie α-částic vyletujících ze vzorku obsahujícím izotop $^{239}$Pu a příměs izotopu $^{238}$Pu a porovnejte je s tabelovanými hodnotami. Stanovte relativní zastoupení izotopu $^{238}$Pu ve vzorku s přesností lepší než 10 \%, jsou-li $t_½({}^{238}\text{Pu}) = 87.71 \U{yr}$ a $t_½({}^{239}\text{Pu}) = 24.13 \cdot 10^3 \U{yr}$.
\end{enumerate}

\section{Teoretická část}
V práci budeme měřit α-záření pomocí křemíkového α-spektrometru. Spektrometr měří počet signálů spadajících do každého \textit{kanálu} $n(\mathrm{ch.})$, číslo kanálu je přímo úměrné kinetické energii $T$ příchozí částice. Počet detekovaných částic je zatížený chybou, která se rovná $\sigma(n) = \sqrt{n}$ (protože se jedná o Poissonovo rozložení). Detektor kalibrujeme tak, že naměříme spektrum známého prvku a kanálu, ve kterém se nachází těžiště naměřeného píku, přiřadíme předpokládanou energii. Pro energetické rozlišení kalibrovaného přístroje platí vztah [1]
\begin{equation}
    \sigma(T) = \frac{\Gamma}{2 \sqrt{N \, 2 \ln 2}} \: ,
    \label{eq-energie-neurcitost}
\end{equation}
kde $\Gamma$ je \textit{pološířka} píku (tj. plná šířka v polovině maxima, FWHM) a $N$ je celkový počet detekovaných impulzů – tedy součet $n$ jednotlivých kanálů.

\bigskip

Částice ze vzorku vylétávají izotropně do všech stran, my je ovšem zachytáváme pouze na detektoru o relativně malém povrchu. Celkovou aktivitu vzorku můžeme ovšem dopočítat pomocí vzorce [1]
\begin{align}
    A &= \frac{4\pi \, \ell^2 \, N}{S \, t} \: ,
    &
    \eta(A) &= \sqrt{ \eta(N)^2 + 4\eta(\ell)^2 }
    = \sqrt{ N^{-1} + 4\eta(\ell)^2 } \: ,
    \label{eq-aktivita}
\end{align}
kde $\ell$ je vzdálenost mezi vzorkem a detektorem, $S$ je plocha detektoru, $N$ je celkový počet naměřených impulzů, $t$ je celkový čas měření a $\eta(X) = \sigma(X) / X$ je relativní chyba veličiny $X$.

\bigskip

Pro záření o energii mezi 4 a 7 MeV platí pro ztrátu kinetické energie ve vzduchu za normálních podmínek empirická formule [1]:
\begin{equation}
    -\frac{\text d T}{\text d x} = \frac{2}{3} \, \frac{1}{\xi \sqrt T} \: ,
    \label{eq-empiricka}
\end{equation}
kde $x$ je vzdálenost od zdroje a $\xi = 0.31 \U{cm \cdot MeV^{-3/2}}$. V našem experimentu nebudeme měnit vzdálenost detektoru, ale tlak – bude se tedy měnit \textit{efektivní tloušťka vzduchového sloupce} $x_{\text{eff.}} = \ell \, p / p_{\text{atm.}} \:$. Naším cílem bude ověřit platnost formule \eqref{eq-empiricka} pomocí numerické derivace:
\begin{align}
    T'(x_{\text{eff.}})
    = „\frac{\Delta T\;}{\Delta x_{\text{eff.}}\!\!}“
    = \frac{T_{j+1} - T_j}{x_{j+1} - x_j}
    \label{eq-deriv}
\end{align}

\bigskip

V úloze budeme také chtít určit poměr zastoupení dvou izotopů plutonia ve vzorku ze znalosti jejich poločasů rozpadu a počtu naměřených impulzů. K tomu nám poslouží rozpadový zákon:
\begin{align*}
    \frac{\text{d}N}{\text{d}\,t} = \frac{\ln 2}{t_½} \, M \: ,
\end{align*}
kde $N$ je počet rozpadů, $t_½$ je poločas rozpadu a $M$ je počet zbývajících atomů daného izotopu. Protože pracujeme na časových škálách o mnoho řádů menších, než jsou poločasy rozpadu jednotlivých izotopů, můžeme $M$ považovat za konstantní a symbolicky rovnici „zintegrovat“:
\begin{align*}
    N = \frac{\ln 2}{t_½} \, M \, t \: ,
\end{align*}
kde $N$ je počet rozpadů naměřený za dobu $t$. Skládá-li se vzorek ze dvou izotopů A a B, relativní zastoupení izotopu A získáme snadno jako:
\begin{align}
    R(\text A)
    &= \frac{M(\text A)}{M(\text A) + M(\text B)}
    = \frac{N(\text A)}{\, N(\text A) +  \dfrac{t_½(\text B)}{t_½(\text A)} \, N(\text B) \,}
    \: ,
    &
    \eta(R\small(A\small))
    = \frac{t_½(\text B)}{t_½(\text A)}
    \sqrt{\frac{N(\text B)}{N(\text A)^2} + \frac{N(\text B)^2}{N(\text A)^3}}
    \: .
    \label{eq-relativni-zastoupeni}
\end{align}
Relativní chybu $R$ lze určit přímo z pravidla o „šíření variance“, tedy:
\begin{equation*}
    \sigma(X)^2 = \sum_j \left(\frac{\partial X}{\partial y_j} \right)^2 \sigma(y_j)^2 \: ,
\end{equation*}
a následnou úpravou.

\section{Výsledky měření}

\subsection{Kalibrace, rozlišení a aktivita vzorku}
Uvedená plocha detektoru je ${900 \U{mm^2}}$, vzorek $^{241}$Am jsme umístili do vzdálenosti ${\ell = 36.4 \pm 0.5 \U{mm}}$ od něj. Poté jsme měřili po dobu $t=116.2 \U{s}$. Naměřili celkem $N = 4.017 \cdot 10^4$ impulzů a těžiště naměřeného píku identifikovali s očekávanou energií $T = 5.4857 \U{MeV}$, šířka píku byla $\Gamma = 145.9 \U{keV}$. Podle \eqref{eq-energie-neurcitost} vypočítáme rozlišení kalibrovaného přístroje:
\begin{equation*}
    \sigma(T) = \frac{\Gamma}{2 \sqrt{N \, 2 \ln 2}}
    = 0.31 \U{keV}
\end{equation*}

Dále z týchž dat podle \eqref{eq-aktivita} vypočteme celkovou aktivitu vzorku:
\begin{gather*}
    A = \frac{4\pi \, \ell^2 \, N}{S \, t} \\
    \eta(A) = \sqrt{N^{-1} + 4 \eta(\ell)^2}
\end{gather*}
\begin{equation*}
    A = (6400 \pm 1500) \U{s^{-1}}
\end{equation*}

\subsection{Měření s proměnným tlakem}
Dále jsme detektor i vzorek schovali pod hrnec a vývěvou odčerpali všechen vzduch. Poté jsme postupně zvyšovali tlak a proměřovali spektrum.

\phantom{.}
\begin{minipage}{\linewidth}
    \centering
    \begin{tabular}{ rl|rl }
        \multicolumn{2}{c}{$p [\U{10^{-3} atm.}]$} &
        \multicolumn{2}{c}{$T [\U{MeV}]$}
        \csvreader[
            head to column names,
            separator=comma
        ]{data.csv}{}
        {
           \csviffirstrow{\\\hline}{\\}
           \tlak &$\pm$ 10 & \energy & $\pm$ \sigmaenergy
        }
    \end{tabular}
    \vspace{\baselineskip}
    \tabcaption{Naměřená závislost $T(p)$}
    \label{energie-tlak}
    \vspace{\baselineskip}
\end{minipage}

\phantom{.}
\begin{minipage}{\linewidth}
    \vspace{\baselineskip}
    \centering
    \def\gptboxheight{15cm}
    \begin{gnuplot}[terminal=epslatex,terminaloptions={color size 15cm, 8cm}]
        set datafile separator ','

        set xlabel '$p [10^{-3} \U{atm.}]$'
        set ylabel '$T [MeV]$'

        set xrange [0:1000]
        set yrange [0:7]

        a = b = c = 1
        f(x) = a*x**2 + b*x + c
        fit f(x) 'data.csv' skip 1 using 1:4:5 yerrors via a,b,c

        plot 'data.csv' skip 1 using 1:4:5 w yerror t 'naměřená data', f(x) t 'fit polynomem 2. stupně'
    \end{gnuplot}
    \figcaption{Graf naměřené závislosti energie na tlaku v hrnci.}
    \label{graf-basic}
    \vspace{\baselineskip}
\end{minipage}

Následně jsme podle \eqref{eq-deriv} vypočetli numerickou derivaci těchto dat. Protože jednomu bodu grafu $\Delta T / \Delta x$ „náleží“ dva různé body $p,T$, využili jsme průměru $(T_j + T_{j+1})/2$.

\phantom{.}
\begin{minipage}{\linewidth}
    \vspace{\baselineskip}
    \vspace{\baselineskip}
    \centering
    \begin{tabular}{ rl|rl|rl }
        \multicolumn{2}{c}{$x [\U{mm}]$} &
        \multicolumn{2}{c}{$T [\U{MeV}]$} &
        \multicolumn{2}{c}{$-\frac{\Delta T}{\Delta x} \; [\U{keV/m}]$}
        \\[5pt]\hline
        \csvreader[
            head to column names,
            separator=comma
        ]{data-deriv.csv}{}
        {
           \csviffirstrow{}{\\}
           \pos &$\pm$ \poserr & \energ & $\pm$ \energerr & \deriv & $\pm$ \deriverr
        }
    \end{tabular}
    \vspace{\baselineskip}
    \tabcaption{Numerická derivace}
    \label{energie-tlak}
    \vspace{\baselineskip}
\end{minipage}

\phantom{.}
\begin{minipage}{\linewidth}
    \vspace{\baselineskip}
    \centering
    \def\gptboxheight{15cm}
    \begin{gnuplot}[terminal=epslatex,terminaloptions={color size 15cm, 8cm}]
        set datafile separator ','

        set xlabel '$T [MeV]$'
        set ylabel '$-\Delta T / \Delta x \; [keV / m]$'

        set xrange [2:5.5]
        #set yrange [0:7]

        xi = 0.31
        f(T) = 2/(10*3*xi*sqrt(T))

        plot 'data-deriv.csv' skip 1 using 3:5:6 w yerror t 'numerická derivace dat', f(x) t 'empirický vztah $\frac{2}{3\xi\sqrt T}$'
    \end{gnuplot}
    \figcaption{Přesnost numerické derivace je žalostná.}
    \label{graf-deriv}
    \vspace{\baselineskip}
\end{minipage}

Tohle je ten okamžik, kdy se autorovi před očima začaly zjevovat všechny jeho životní chyby. Jen se podívejte na graf v obrázku \ref{graf-deriv}, je to kolosální propadák! Zcela jistě existují i robustnější způsoby výpočtu numerické derivace, autorovi bohužel nejsou známé. Nenechme se ovšem tímto dílčím neúspěchem odradit a pokračujme k měření izotopů plutonia.

\subsection{Izotopy plutonia}
Při měření směsi izotopů $\,{}^{238}$Pu\;\,a $\,{}^{239}$Pu po dobu $t=1161 \U{s}$ jsme dostali spektrum obsahující dva píky. Pík s vyšší energií náleží $^{238}$Pu a je tvořen $N({}^{238}\text{Pu}) = 1\,326$ impulzy. Druhý pík má $N({}^{239}\text{Pu}) = 217\,351$ impulzů. Četnosti dosadíme do vzorce \eqref{eq-relativni-zastoupeni}:
\begin{equation*}
    R({}^{238}\text{Pu})
    = \frac{N({}^{238}\text{Pu})}{\, N({}^{238}\text{Pu}) +  \dfrac{t_½({}^{239}\text{Pu})}{t_½({}^{238}\text{Pu})} \, N({}^{239}\text{Pu}) \,}
    = 2.22 \cdot 10^{-5}
\end{equation*}
Dosazením do vzorce pro chybu $R$ ovšem dostaneme nerealisticky vysoké číslo:
\begin{equation*}
    \eta\big(R({}^{238}\text{Pu})\big)
    = \frac{t_½({}^{239}\text{Pu})}{t_½({}^{238}\text{Pu})}
    \sqrt{\frac{N({}^{239}\text{Pu})}{N({}^{238}\text{Pu})^2} + \frac{N({}^{239}\text{Pu})^2}{N({}^{238}\text{Pu})^3}}
    = 2 \cdot 10^{-2}
\end{equation*}
Naměřené energie píků jsou:
\begin{align*}
    T({}^{238}\text{Pu}) &= 5520.5 \pm 0.9 \U{keV} \\
    T({}^{239}\text{Pu}) &= 5162.5 \pm 0.1 \U{keV}
\end{align*}



\section{Diskuse}
Při ověřování empirické formule pro úbytek energie byla použita numerická metoda, která není robustní, tzn. je velmi náchylná na šum ve vstupních datech. Lepšího výsledku bychom dosáhli například tak, že bychom proložili křivku původními daty a tuto analyticky derivovali. Bylo by ovšem potřeba vypočítat konfidenční oblast takové křivky.

Vypočtená chyba relativního zastoupení $R$ vychází nerealisticky velká, k jejímu výpočtu byl přitom použit vzorec pro šíření chyby, který alespoň pro gaussovsky distribuované malé chyby platí, a pozorování, že pro počet impulzů platí $\sigma(N) = \sqrt{N}$ (tj. je poissonovsky rozložený). K takto zkreslenému výsledku jsme došli buďto proto, že součet všech impulzů v píku už není rozložený poissonovsky, nebo proto, že použitý vzorec pro šíření chyby neplatí pro poissonovská data ani přibližně. Ať tak nebo tak, autor je výsledkem upřímně velmi překvapen, a to i přesto, že se ve fyzikálním praktiku takové zábavné nesrovnalosti nevyskytují zrovna zřídka.


\section{Závěr}
Podařilo se kalibrovat α-spektrometr, jeho rozlišení je
\begin{equation*}
    \sigma(T) = 0.31 \U{keV} \: .
\end{equation*}
Dále jsme určili aktivitu kalibračního vzorku $^{241}$Am jako:
\begin{equation*}
    A = (6400 \pm 1500) \U{s^{-1}}
\end{equation*}
Podařilo se proměřit závislost $T(p)$ a porovnat úbytek energie s empirickým vztahem. Ačkoliv empirický vztah řádově odpovídal naměřeným výsledkům, data bohužel obsahovala příliš šumu na to, abychom pozorovali empirickou závislost.
\\
Podařilo se naměřit energie dvou izotopů plutonia
\begin{align*}
    T({}^{238}\text{Pu}) &= 5520.5 \pm 0.9 \U{keV} \\
    T({}^{239}\text{Pu}) &= 5162.5 \pm 0.1 \U{keV}
\end{align*}
a vypočítat, že relativní zastoupení izotopu $^{238}$Pu ve vzorku je
\begin{equation*}
    R({}^{238}\text{Pu})
    = 2.22 \cdot 10^{-5} \: .
\end{equation*}


\section{Literatura}
[1] Nosek D. a J. Vrzal. Spektrometrie záření α. Dostupné z: \url{https://physics.mff.cuni.cz/vyuka/zfp/_media/zadani/texty/txt_405.pdf}. 30. září 20031
\\
\st{[2] Nesnáším BibTeX.}


\end{document}